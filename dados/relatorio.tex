\documentclass[12pt,a4paper]{article}

% Pacotes essenciais ABNT
\usepackage[utf8]{inputenc}
\usepackage[T1]{fontenc}
\usepackage[brazil]{babel}
\usepackage{setspace}
\usepackage{graphicx}
\usepackage{indentfirst}
\usepackage{geometry}
\usepackage{hyperref}
\usepackage{float}
\usepackage{titlesec}
\usepackage{caption}
\usepackage{times} % Fonte Times New Roman

% Margens e espaçamento conforme ABNT (NBR 14724)
\geometry{
    left=3cm,
    right=2cm,
    top=3cm,
    bottom=2cm
}
\onehalfspacing % Espaçamento 1,5
\setlength{\parindent}{1.25cm} % Recuo de parágrafo ABNT
\setlength{\parskip}{0pt} % Sem espaço entre parágrafos

% Remover "Capítulo" e ajustar espaçamento entre seções
\titleformat{\section}{\bfseries\normalsize\uppercase}{\thesection}{1em}{}
\titleformat{\subsection}{\bfseries\normalsize}{\thesubsection}{1em}{}
\titlespacing*{\section}{0pt}{1.5em}{1em}
\titlespacing*{\subsection}{0pt}{1em}{0.5em}

% Configuração de links
\hypersetup{
    colorlinks=true,
    linkcolor=black,
    urlcolor=blue,
    citecolor=black
}

\begin{document}

% CAPA (modelo ABNT adaptado)
\begin{titlepage}
    \centering
    {\large \textbf{CENTRO PAULA SOUZA}}\\
    {\large \textbf{FACULDADE DE TECNOLOGIA DE SANTOS – FATEC RUBENS LARA}}\\[4cm]

    {\bfseries\LARGE SISTEMA DE BUSCA E ANÁLISE DE SIMILARIDADE ENTRE ARTIGOS CIENTÍFICOS COM TF-IDF E PCA}\\[3cm]

    {\large \textbf{JOÃO PEDRO DIAS}}\\[4cm]

    {\large Santos – SP}\\
    {\large \today}
\end{titlepage}

% SUMÁRIO
\renewcommand{\contentsname}{Sumário}
\tableofcontents
\newpage

% INTRODUÇÃO
\section*{Introdução}
\addcontentsline{toc}{section}{Introdução}
O presente trabalho tem como objetivo desenvolver um sistema de busca e análise de similaridade entre artigos científicos utilizando técnicas de Processamento de Linguagem Natural \textit{(PLN)}. Foram aplicados conceitos de vetorização de textos com \textit{TF-IDF}, cálculo de similaridade de cosseno, redução de dimensionalidade por \textit{PCA} e visualização tridimensional dos vetores. Como resultado, o sistema é capaz de identificar os artigos mais semelhantes a uma consulta textual (\textit{query}) e exibir suas similaridades, ângulos e resumos, além de representá-los graficamente em um espaço 3D.

% DESCRIÇÃO DO DATASET
\section{Descrição do Dataset}
O conjunto de dados utilizado foi obtido no \textit{Kaggle}, no repositório \textit{"Topic Modeling for Research Articles"}. O dataset contém milhares de artigos científicos, cada um com título, resumo e o tema relacionado. Neste projeto, foram utilizados os campos de título e resumo, sendo aplicadas técnicas de pré-processamento para limpeza textual e extração de termos relevantes para a análise de similaridade.

% ESTRUTURA E FUNCIONAMENTO
\section{Estrutura e Funcionamento do Sistema}
O sistema foi implementado em Python, em um ambiente do Google Colab e estruturado em diferentes etapas. A seguir, cada parte do código é explicada com base nos principais blocos apresentados no \textit{notebook}.

\subsection{Importação das Bibliotecas}
Nesta etapa, foram importadas as bibliotecas essenciais para o desenvolvimento do sistema, incluindo:
\begin{itemize}
    \item Manipulação de dados: \textit{pandas}, \textit{numpy};
    \item Processamento de texto: \textit{nltk}, \textit{re};
    \item Cálculo de TF-IDF e similaridade de cosseno: \textit{scikit-learn};
    \item Visualização dos vetores: \textit{matplotlib}.
\end{itemize}

Além disso, foi realizado a instalação dos recursos do \textit{NLTK}, como o \textit{tokenizador} e as \textit{stopwords}.

\begin{figure}[H]
    \centering
    \includegraphics[width=0.9\textwidth]{01_importacao_bibliotecas.png}
    \caption{Importação das bibliotecas e instalação dos recursos do NLTK.}
\end{figure}

\subsection{Carregamento e Preparação dos Dados}
O conjunto de dados foi carregado, contendo as colunas \textbf{TITLE} e \textbf{ABSTRACT}. Em seguida, foi criada uma nova coluna denominada \textbf{TEXT}, que une ambas as informações, ampliando o contexto disponível para o cálculo de similaridade.

\begin{figure}[H]
    \centering
    \includegraphics[width=0.9\textwidth]{02_carregamento_dataset.png}
    \includegraphics[width=0.9\textwidth]{carregamento_e_visu_data_set.png}
    \includegraphics[width=0.9\textwidth]{concat_colunas.png}
    \caption{Carregamento do dataset e criação de nova coluna \texttt{TEXT}.}
\end{figure}

\subsection{Pré-processamento dos Textos}
Foram aplicadas transformações para limpar e normalizar os textos, facilitando o cálculo de similaridade. As etapas incluem: conversão para minúsculas, remoção de pontuação (mantendo hífen), eliminação de espaços extras, \textit{tokenização}(divide o texto em palavras ), remoção de \textit{stopwords}(elimina palavras comuns) e \textit{stemming}(reduz palavras à sua forma base).

\begin{figure}[H]
    \centering
    \includegraphics[width=0.9\textwidth]{03_preprocessamento.png}
    \caption{Aplicamento da função de pré-processamento dos textos.}
\end{figure}

\subsection{Aplicação do TF-IDF}
Após o pré-processamento, aplicou-se o método TF-IDF (Term Frequency–Inverse Document Frequency), que transforma cada documento em um vetor numérico, permitindo mensurar a relevância dos termos.

\begin{figure}[H]
    \centering
    \includegraphics[width=0.9\textwidth]{04_calculo_matrix_tfidf.png}
    \caption{Vetorização dos textos utilizando TF-IDF.}
\end{figure}

\subsection{Função de Busca e Visualização 3D}
A função implementada realiza a busca de artigos com base em uma \textit{query}, utilizando TF-IDF e similaridade de cosseno, além de gerar uma visualização tridimensional interativa dos vetores.

\begin{figure}[H]
    \centering
    \includegraphics[width=0.9\textwidth]{05_funcao_busca_artigo_pt1.png}
    \includegraphics[width=0.9\textwidth]{06_funcao_busca_artigo_pt2.png}
    \caption{Função principal de busca e geração de visualização 3D.}
\end{figure}

\subsection{Execução de Consultas Interativas}
Por fim, foi implementado um \textit{loop} que permite ao usuário realizar consultas diretamente no terminal. O programa é encerrado quando o usuário digita "sair", "exit" ou "quit".

\begin{figure}[H]
    \centering
    \includegraphics[width=0.9\textwidth]{07_interface_terminal.png}
    \caption{Loop interativo para consultas.}
\end{figure}

% RESULTADOS
\section{Resultados}
A seguir, são apresentados exemplos de consultas realizadas no sistema e suas respectivas saídas.

\begin{figure}[H]
    \centering
    \includegraphics[width=0.9\textwidth]{08_resultado1_da_busca.png}
    \caption{Resultados da busca 1 (consulta: \textit{hyperparameter tuning machine learning models}).}
\end{figure}

\begin{figure}[H]
    \centering
    \includegraphics[width=0.9\textwidth]{09_resultado2_da_busca.png}
    \caption{Resumo do artigo mais relevante.}
\end{figure}

\begin{figure}[H]
    \centering
    \includegraphics[width=0.9\textwidth]{10_resultado3_da_busca.png}
    \caption{Visualização 3D dos vetores da consulta e artigos mais similares.}
\end{figure}

Observa-se que os resultados retornaram artigos relacionados à otimização de hiperparâmetros, apresentando valores de similaridade entre 0.38 e 0.48. O resumo exibido demonstra coerência temática com a consulta, reforçando a eficácia do modelo em identificar artigos semanticamente relevantes.

\begin{figure}[H]
    \centering
    \includegraphics[width=0.9\textwidth]{11_resultado1_da_busca2.png}
    \caption{Resultados da busca 2 (consulta: \textit{regularization lasso linear regression}).}
\end{figure}

\begin{figure}[H]
    \centering
    \includegraphics[width=0.9\textwidth]{12_resultado2_da_busca2.png}
    \caption{Resumo do artigo mais relevante.}
\end{figure}

\begin{figure}[H]
    \centering
    \includegraphics[width=0.9\textwidth]{13_resultado3_da_busca2.png}
    \caption{Visualização 3D dos vetores da consulta e artigos mais similares.}
\end{figure}

Na segunda busca, o sistema apresentou resultados consistentes com o tema proposto, retornando artigos sobre regularização e métodos \textit{Lasso} em regressões lineares. As visualizações tridimensionais mostraram a proximidade espacial entre os vetores, indicando que o \textit{PCA} manteve uma boa separabilidade sem perda significativa de coerência semântica.

% CONCLUSÃO
\section*{Conclusão}
\addcontentsline{toc}{section}{Conclusão}
O sistema desenvolvido demonstrou a aplicabilidade do TF-IDF na representação numérica de textos e a eficiência da similaridade de cosseno na identificação de artigos semanticamente relacionados.  
A combinação dessas técnicas, aliada à visualização tridimensional, proporcionou uma análise intuitiva e fundamentada das relações entre documentos científicos, evidenciando o potencial das abordagens de Processamento de Linguagem Natural para a organização e recuperação de informações em bases textuais extensas.

\end{document}